\documentclass{article}
\title{How to Modify the Event Lineup on the Valpo ACM Website}
\date{}

\usepackage{minted}

\begin{document}
	\maketitle
	
	\begin{Large}
		\section{Why Would events.json Need Modified?}
	\end{Large}
	Out of all the JSON files, this one will be modified the most often. This JSON file contains the data for each event held by Valpo ACM. 
	
	\vspace{48pt}
	\begin{Large}
		\section{General Formatting}
	\end{Large} 
	Although the formatting for this JSON file is fairly straightforward, there are some fields that require some explanation to accurately populate this file. Let's take a look at each property to understand what each thing does.
	
	\begin{minted}[breaklines, breakanywhere, frame=lines, framesep=5mm, linenos=true, tabsize=2]{json}
		{
			"term": "Spring 2026",
			"events": [
				{
					"title": "Fruitful Voids and Gardens of Forking Paths: Interactive Narratives in Games",
					"guest": "Featuring Professor Buinicki",
					"location": "TBD",
					"date": "TBD",
					"time": "TBD",
					"description": "Let's discuss similarities and differences between traditional storytelling and game narratives, including the importance of branching storylines and player agency."
				}
			]
		}
	\end{minted}
	
	\subsection{"title"}
	The "title" property is one of the easiest of them all to comprehend. This holds the title of the event. 
	
	\subsection{"guest"}
	The "guest" property is an \textit{optional} value that varies on events. In the example above, there is a guest giving a discussion. In other circumstances, this property can be omitted. A couple cases where you can use this are:
	\begin{itemize}
		\item For a Guest Speaker (ex: Featuring ...)
		\item An event presented by another organization (ex: Presented by ...)
		\item Another organization co-hosting the event with Valpo ACM (ex: Co-hosted by ...)
		\item Any combination of these together (ex: Featuring ... presented by ...)
	\end{itemize}
	It is important to note that the examples above can use different (and possibly better) words.
	
	\subsection{"location"}
	The "location" property is the location of the event. For the purposes of the website, you can put an abbrevation of the building and room number (ex: Gellersen 125 would be GEM 125) . In cases where the location is not finalized yet, it is acceptable to put "TBD" in this field. Just be sure to modify this property later for visitors of the website.
	
	\subsection{"date"}
	The "date" property contains the month and day an event is held on. This is because there is logic in the backend of the website to assign the year properly based on the term property shown in the example above. It is acceptable to enter the date as "Jan 1" or "January 1" for example. \textbf{Do not enter the date as 1/1 since this will cause issues.} Once again, if a date is not set for the event yet, it is okay to enter "TBD" in this property.
	
	\subsection{"time"}
	The "time" property holds the time the event is held on. An example of proper formatting is "6:00 PM". Do not enter the time any other way as it will break backend logic that powers some features of the website.  
	
	\subsection{"description"}
	The "description" property contains the description of the event. The description can be as lengthy or short as needed for the event. If there is not a description ready, this field can be omitted or left blank. 
	
	\subsection{"slides"}
	The "slides" property contains a link for any resources used during the event that can be accessed following the event. When adding a link, make sure the link is viewable on Google Drive since in most cases the link will be from the Valpo ACM Google Drive folder. 
\end{document}