\documentclass{article}
\title{How to Modify the Projects on the Valpo ACM Website}
\date{}

\usepackage{minted}

\begin{document}
	\maketitle
	
	\begin{Large}
		\section{Why Would projects.json Need Modified?}
	\end{Large}
	This file would only require modification for a couple of reasons. One of them is to add new projects that Valpo ACM has worked on. The other is to modify data on an existing entry. Besides those two reasons, this file \textit{should not} be modified very often. 
	
	\vspace{48pt}
	\begin{Large}
		\section{General Formatting}
	\end{Large} 
	
	\begin{minted}[breaklines, breakanywhere, frame=lines, framesep=5mm, linenos=true, tabsize=2]{json}
		{
			"projects": [
				{
					"title": "Valpo ACM Game",
					"contributors": [
						{
							"name": "Eric Yager"
						},
						{
							"name": "Ethan Hawk"
						},
						{
							"name": "Nathan Harmon"
						},
						{
							"name": "Alex Luke"
						},
						{
							"name": "Spencer Gannon"
						},
						{
							"name": "Matthew Spivey"
						},
						{
							"name": "Jace Nowacki"
						}
					],
					"description": "A simplistic top-down shooter game based on retro games like Space Invaders.",
					"github": "https://github.com/valpo-acm/valpo-acm-game"
				}
			]
		}
	\end{minted}
	
	\subsection{"title"}
	This property is the title of the project. This should typically align with the name of the corresponding GitHub repository. 
	
	\subsection{"contributors"}
	This property is an array of entries holding the name of each contributor to the project. This object may be quite large depending on factors in regards to the project. To get a general idea on how it is formatted, consult the example above. 
	
	\subsection{"description"}
	This property is the description of the project. Unlike the description property in \textit{activity.json}, this description property can be lengthy and informative. The example above is an outlier in this sense.
	
	\subsection{"github"}
	This property contains the link to the project's GitHub repository. That's about it for this property. 
\end{document}